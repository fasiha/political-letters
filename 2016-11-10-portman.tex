\documentclass{letter}
\usepackage{hyperref}
\addtolength{\topmargin}{-1.5in}
\addtolength{\textheight}{2.0in}


\signature{Ahmed \& Emily}
\address{\input{address.tex}}
\begin{document}


\begin{letter}{
  The Honorable Robert Portman \\
  37 West Broad Street \\
  Room 300 \\
  Columbus, OH 43215}

\opening{Dear Senator Portman,}

We live in Dayton, and this is the first time we've written to you, but please
accept our congratulations on winning yesterday's election, and also our thanks
for your continued service to Ohioans.

Starting from the primaries and through the Presidential campaign, all
candidates including President-Elect Trump decried the influence of money on
politics. For example, that 158 families donate half the campaign money raised
in America, or that Congresspeople spend between 30\% and 70\% of their time
raising money.

This is distressing, both because it suggests Congresspeople are more sensitive
to the interests of donors, but also because of the wastefulness of spending
{\em so} much time fundraising---time that could be spent reading or writing
legislation, or networking with constituents or colleagues.

What do you make of these problems? What is your perception of the role wealthy
donors have on your activities as Senator, and the cost of fundraising?

We have these questions because we've heard about public funding of political
campaigns, a system where instead of the wealthy or super-PACs, Congressional
candidates raise money from the public. Small donations by constituents to
candidates are matched, as much as six-fold, by a public fund. This enlarges the
pool of local donors, reduces the time spent calling national donors, and
increases citizen interest and participation in politics. Maine and New York
City have state and city versions of this idea that have positively impacted
elections there.

Bills to bring such programs to the federal level have been introduced. E.g.,
H.R. 20, John Sarbanes' Government By the People Act, enjoys wide support in the
House. What do you think about such proposals? Do you have other ideas on how to
address the influence of money in politics, which all candidates have decried?
We understand that publicly-funded elections would constitute yet more spending
by the government, but of all the spending done by the government, reducing the
impact of money on politics seems to be the most wholesome.

Many thanks for your thoughts, and again, congratulations.


\closing{Very respectfully,}


\end{letter}
\end{document}
