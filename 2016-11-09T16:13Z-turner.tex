\documentclass{letter}
\usepackage{hyperref}
\addtolength{\topmargin}{-1.5in}
\addtolength{\textheight}{2.0in}


\signature{Ahmed \& Emily}
\address{\input{address.tex}}
\begin{document}


\begin{letter}{
  The Honorable Michael Turner \\
  120 West 3rd St, Ste. 305 \\
  Dayton, OH 45402}

\opening{Dear Mr Turner,}

We live in Oakwood, and this is the first time we've written to you, but please
accept both our congratulations---on winning Ohio's tenth district
yesterday---and thanks, for your service to us.

Over the electyion cycle, we learned some disturbing facts about the influence
of money on Congress, including from President-Elect Trump. For example, that
158 families donate half the campaign money raised in America. Or that
Congresspeople spend between 30\% and 70\% of their time raising money.

The understanding is that this funding system makes Congresspeople more
sensitive to donors than to constituents. Is this understanding correct? Dayton
has a concentration of defense contractors, large and small, with wealthy
executives and managers, some of whom must be active political donors. Could you
help us understand the role wealthy donors play in your activities as our
Representative?

Furthermore, the idea that you and your fellow Congresspeople spend {\em so}
much time fundraising is troubling just because that's time not spent reading
and writing legislation, nor networking with constituents and colleagues and
staffers. Could you also give us a sense of how burdensome fundraising is?

We ask these two questions because we've heard about public funding of political
campaigns, a system where instead of the wealthy or super-PACs, Congresspeople
raise money for campaigns from the public. Small donations by constituents to
candidates could be matched, as much as six-fold, by a public fund, thereby
enlarging the pool of donors Congress pursues as well as reducing the time spent
pursuing them. Your colleague, John Sarbanes from Maryland, has introduced a
bill, H.R. 20, the Government by the People Act, which seems like a promising
attempt to address these problems. It's backed by 150 Representatives.

We wanted to get your insight into this idea and this bill. We understand it'd
be yet another governmental expense, but of all the things that the government
spends money on, addressing the influence of money on Congress seems like the
most worthy.

Many thanks for your thoughts, and again, congratulations.

\closing{Very respectfully,}


\end{letter}
\end{document}
