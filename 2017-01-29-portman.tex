\documentclass{letter}
\usepackage{hyperref}
\addtolength{\topmargin}{-1.5in}
\addtolength{\textheight}{2.0in}


\signature{Ahmed \& Emily}
\address{\input{address.tex}}
\begin{document}


\begin{letter}{
  The Honorable Robert Portman \\
  37 West Broad Street \\
  Room 300 \\
  Columbus, OH 43215}

\opening{Dear Senator Portman,}

Thank you for your letter dated December 9, 2016, responding to our request that you support campaign finance reform and election reform. We were so disappointed with your equating ``unions, businesses, interest groups'' with ``individuals'' when it came to these pressing matters. We understand that the law, both as written by Congress and as interpreted by the Supreme Court, calls for equality of these disparate entities in some technical senses, but are you aware that the Supreme Court has repeatedly supported citizen-funded elections and contribution-matching by a public fund? To combat the influence of money on the legislative and executive branches, such legislation on a national level is crucial. Please reconsider your stance on this matter (H.R.20 and S.J.Res.5), because the calls to rein in the corrupting effect of money in politics, made loudly during last year's campaign including by now-President Trump, seem to have completely forgotten.

But we are writing to you today specifically about the behavior of President Trump. We want you to know that his behavior during his first week in office is has left us appalled, flabbergasted, and outraged.

We find it astonishing that science, including much done at our alma mater of Ohio State and other fine Ohio schools, is being discredited and suppressed.

We find it inconceivable that journalists critical of the President are being called on to be fired.

We find it heart-breaking that xenophobia is being promoted in terms of crime prevention.

We find it frightening that the profits of big businesses are being prioritized over the health and welfare of citizens, in the mistreatment of the EPA and USDA.

And we are surprised to see no word from your office, no press release nor tweet, denouncing these outrages. Please let us know what you are planning to do about this suppression of science, this threat to the environment, this slandering of our media, and this oppression of guests of our country, by the highest office of the land.

\closing{Very respectfully,}


\end{letter}
\end{document}
