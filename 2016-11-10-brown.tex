\documentclass{letter}
\usepackage{hyperref}
\addtolength{\topmargin}{-1.5in}
\addtolength{\textheight}{2.0in}


\signature{Ahmed \& Emily}
\address{\input{address.tex}}
\begin{document}


\begin{letter}{
  The Honorable Sharrod Brown \\
  200 North High St. \\
  Room 614 \\
  Columbus, OH 43215}

\opening{Dear Senator Brown,}

We live in Dayton, and this is the first time we've written to you, but please
accept our thanks for your service to Ohioans.

Throughout this Presidential campaign, we've heard many disturbing facts about
the influence of money in politics. Senator Sanders discussed this at length, as
has Secretary Clinton. President-Elect Trump also decried this, in his own
unique way. For example, that 158 families donate half the campaign money raised
in America, or that Congresspeople spend between 30\% and 70\% of their time
raising money.

Such news is distressing. First, it suggests that Congresspeople are beholden to
donors, and thus more sensitive to their interests. And second, the time burden
seems truly wasteful. Instead of reading or writing legislation, or networking
with colleagues or constituents, we picture legislators calling donors or
attending fundraisers.

We'd like to get your input on these issues. Can you help us understand the role
wealthy donors play in your activities as Senator, and the cost fundraising
imposes on you? Is it as bad as Bernie and Hillary and Donald say, or worse?

We have these questions because we've heard about public funding of political
campaigns, a system where instead of the wealthy or super-PACs, candidates raise
money from the public. Small donations by constituents to candidates could be
matched, as much as six-fold, by a public fund, thereby enlarging the pool of
donors Congress pursues as well as reducing the time spent pursuing them.
Candidates who engage with such programs would agree to limits on private
funding.

In Maine as well as in New York City, public financing of elections has been
received very positively. Candidates who participate in these can and do
overcome candidates using private funding. Do you think such systems can be
brought to the federal level? Representative John Sarbanes' Government By the
People Act seeks to do just this.

Your thoughts on these matters would be much appreciated.

\closing{Very respectfully,}


\end{letter}
\end{document}
